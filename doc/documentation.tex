\documentclass[11pt,a4paper]{article}
\usepackage[utf8]{inputenc}
\usepackage[czech]{babel}
\usepackage[T1]{fontenc}
\usepackage{amsmath}
\usepackage{amsfonts}
\usepackage{url}
\usepackage{amssymb}
\usepackage[left=2cm,right=2cm,top=2.5cm,bottom=2cm]{geometry}
\author{Stanislav Nechutný}
\begin{document}

	\section{Zdroje}

		\subsection{Netlist}


			\url{http://www.expresspcb.com/ExpressPCBBin/NetlistFileFormat.pdf} \\
			\url{https://en.wikipedia.org/wiki/Netlist}

	\section{Dotazy}

		\subsection{Formát netlistu}

			Varinty

				a) \url{http://www.expresspcb.com/ExpressPCBBin/NetlistFileFormat.pdf}

				b) \url{http://cadlab.cs.ucla.edu/~trio/gdif.html}

		\subsection{Inicializace logických obvodů}

			Jak začínat na začátku simulace?

				a) Veškeré komponenty mají na začátku výstup 0 - vychází z předpokladu, že je vypnutý i vstupní proud do logických obvodů a tak z NOT nemůže jít logická 1.

				b) Před začátkem simulace mají již logické obvody zapnuté napájení a tak z INPUT $\rightarrow$ NOT $\rightarrow$ OUTPUT půjde již na začátku logická 1

	\section{Chyby}

		Cislo chyby	popis
		2		Spatny format netlistu
		3		Netlis obsahuje chyby


\end{document}
