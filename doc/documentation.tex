\documentclass[11pt,a4paper]{article}
\usepackage[utf8]{inputenc}
\usepackage[czech]{babel}
\usepackage[T1]{fontenc}
\usepackage{amsmath}
\usepackage{amsfonts}
\usepackage{url}
\usepackage{amssymb}
\usepackage[left=2cm,right=2cm,top=2.5cm,bottom=2cm]{geometry}
\author{Stanislav Nechutný}
\begin{document}

	\newcommand{\slideRef}[1]{\textit{(IMS slide #1)}}

	\begin{titlepage}

		\begin{center}

			\textsc{
				\Huge
					Vysoké učení technické v~Brně\\
				\huge
					Fakulta informačních technologií
			}\\

			\vspace{\stretch{0.320}}

			\LARGE
					IMS - Modelování a simulace\\
					3. Simulátor číslicových obvodů\\~\\
			\Huge{}
					Dokumentace
			\vspace{\stretch{0.650}}

		\end{center}

		{\Large
			30. listopadu 2015
			\hfill
			Stanislav Nechutný, Miloslav Smutka
		}

	\end{titlepage}


	\section{Úvod a motivaci}
		Tato práce se zaměřuje na tvorbu simulátoru logických obvodů, který implementuje hradla AND, OR, XOR, NOT, NOR, NAND. Pro implementaci je zvolen jazyk C++ a jako vstup jednoduchý textový soubor, jehož formát je odvozen od NETLISTu. Blíže je vstupní formát popsán v této dokumentaci v kapitole \ref{netlist}.
		
		
			
		\subsection{Konzultace a zdroje}
			Chování logických členů v našem modelu systému \slideRef{18} bylo konzultováno s panem Bc. Stanislavem Koskem, který je absolventem bakalářského studia na Fakultě elektronické a komunikačních technologií Vysokého učení technického v Brně a ve volném čase se zabývá návrhem logických obvodů. Jednalo se zejména o zjištění typických zpoždění hradel pro zajištění vhodné implementace systému a dalších specifických vlastností hradel, které je třeba v simulaci zohlednit.
			
		\subsection{Ověřování validity modelu}
			Pro ověření validity modelu \slideRef{37} byla použita stavebnice Voltík III a Voltík II, které obsahují většinu, v tomto modelu implementovaných, logických hradel a je tedy možné sledovat průběh zapojení. Pro vizualizaci dat bylo využito připojení led diod a pro vizualizaci zpoždění zaznamenatelném lidským okem byli využity kondenzátory s různou kapacitou.
			
			Připojením výstupu logických hradel na osciloskop bylo sledováno odpovídající chování, jako u simulovaných a dochází tedy k ekvivalentnímu chování systémů \slideRef{26}.
		
		

	\section{Shrnutí relevantních faktů, zdroje informací}
		...	
		
		\subsection{Použité postupy}
			Proč jsou vhodné
			
			Krok byl zvolen 1 nanosekunda, aby nedošlo k problémům s detekcí změn stavových podmínek \slideRef{269}. Jak již víme například z předmětu Informační systémy, tak vzorkovací frekvence by měla být alespoň dvojnásobná pro vhodné vzorkování na diskrétní hodnoty.
			
			Použitým modelem signálů je dvouhodnotový \slideRef{277}, který je výrazně rychlý. 
			
			Bylo zvoleno přiřaditelné zpoždění \slideRef{278} pro každý typ logického hradla. Tato varianta vychází ze zadání, ale je i zcela logickým požadavkem, které vychází z rozdílné vnitřní implementace hradel. Jsou uvnitř použité různé přechody v různém množství a tak samozřejmě dochází k odlišnostem ve zpoždění. Na tento fakt jsme byli také upozornění při konzultaci s Bc. Stanislavem Koskem.
			
			Pro řešení zpětné vazby obvodů bylo zvoleno iterační řešení. Inicializace modelu \slideRef{279} vychází ze stavu, kdy v čase T=0 je sepnut nejen vstup, ale i napájení logických členů, takže k nastavení jejich výstupů dochází v počáteční iteraci. Logické hradlo NOT tedy v čase T=0 bude mít již na výstupu logickou 1, ale pokud má na vstup připojenu logickou 1, tak na výstupu se logická 0 objeví až v čase T=SpožděníHradla. Obdobné to bude u dalších negovaných hradel, jako například NAND apod.
		

			
		\subsection{Původ postupů}
			Popsat, že jsme udělali v pátek brainstorming, popsali 2x dveře a v sobotu to podle toho implementovali.
			
			Na základě návrhu a informací v prezentaci \slideRef{276} byla zvolena implementace diskrétního modelu.
	
	\section{Koncepce metody, přístupu, modelu}

		Uzavřený systém \slideRef{30}
		
		Izomorfní systém \slideRef{27}
	
		...	
		
		Model je deterministický \slideRef{32}, jelikož simuluje logické obvody, které mají deterministické chování a nejsou zde náhodné proměnné \slideRef{75}.
		
		Veškeré komponenty mají na začátku výstup 0 - vychází z předpokladu, že je vypnutý i vstupní proud do logických obvodů a tak z NOT nemůže jít logická 1.
	
	\section{Implementace metody, modelu}
		...
		
		Pro implementaci byl zvolen kvaziparalelismus \slideRef{121} pro jeho jednoduchost pro implementaci a omezení počtu chyb vzniklých případnou synchronizací paralelních procesů. Paralelismus by byl obtížně implementovatelný, jelikož chování jednoho členu systému závisí na předchozích výstupech ostatních.
		
		Na základě analýzy zadání a chování logických prvků bylo rozhodnuto, že bude využito konceptu objektově orientovaného programování \slideRef{168}, které nabízí jazyk C++. Každé simulované hradlo bude instancí třídy implementující konkrétní chování. Společné chování je implementováno v abstraktní třídě od které jednotlivá hradla dědí. Dalším typem objektu je vodič, který slouží k propojení vstupů a výstupů s hradly. Opět bude \textit{n} instancí v závislosti na vstupním souboru.
		
		Na základě zjištěných informací bylo rozhodnuto použít kruhových bufferů
		
		...
		
		Argumentem programu je množství požadovaného modelového času \slideRef{21}, který má být simulován. Jelikož je model implementován v prostředí číslicových systémů, tak je čas diskrétní \slideRef{22}.
		
		...		
		
		
		\subsection{Netlist}
			\label{netlist}

			\url{http://www.expresspcb.com/ExpressPCBBin/NetlistFileFormat.pdf} \\
			\url{https://en.wikipedia.org/wiki/Netlist}
	\section{Experimentování}
		...
		
	\section{Závěr}
		...




	\section{Zdroje}

		

	\section{Chyby}

		Cislo chyby	popis
		
		2		Spatny format netlistu
		
		3		Netlis obsahuje chyby


\end{document}
